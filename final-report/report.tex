% Very simple template for lab reports. Most common packages are already included.
\documentclass[a4paper, 11pt]{article}
\usepackage[utf8]{inputenc} % Change according your file encoding
\usepackage{graphicx}
\usepackage{url}

%opening
\title{Distributed Systems, Advanced Course \\ 
		Project Report}
\author{KTH Royal Institute of Technology \\ 
	School of Information and Communication Technology \\
	Student:Fanti Machmount Al Samisti (fmas@kth.se) \\
	Student:Pradeep Perris (weherage@kth.se)}
\date{\today{}}

\begin{document}

\maketitle

\tableofcontents

\clearpage

\section{Introduction}

\textit{The goal of this project is to design and implement a distributed key-value store in Kompics. We have used well known distrubuted abstraction model to achive this task: }

Our model employs a (static) membership protocol. Data is partisioned using a Hashing function and replicated within each membership nodes. The data consitency among the replicas during in both reads and updates are achived with (N,N) Atomic Register.

\section{Design Overview}

\subsection{System Component}
The following figure depicts the overall design of the system.

{\centering\includegraphics[scale = 0.8]{./images/design_overview.png}\par}

\subsection{Node Toplogy}
{\centering\includegraphics[scale = 0.6]{./images/node_setup.png}\par}

\section{System Abstraction and Implmentation}
\textit{The report should not be too long ($\approx$
	2-3 pages).}

\subsection{Perfect Point to Point Link}

\textit{The report should not be too long ($\approx$
	2-3 pages).}

\subsection{Best Effort Broadcast}

\textit{The report should not be too long ($\approx$
	2-3 pages).}

\subsection{(N,N) Atomic Registry}

\textit{The report should not be too long ($\approx$
	2-3 pages).}

\subsection{Reconfiguration}
\textit{The report should not be too long ($\approx$
	2-3 pages).}

\section{System Simulations and Scenarios}

\subsection{Perfect Point to Point Link}
Perfect Point to Point Link is the base abstraction in our System Model. The properties of Perfect Point to Point Link are;
 
\begin{itemize}
	\item Reliable Delivery: If a correct process p send a message m to correct process q then q eventually delivers m.
	
	\item No Duplication: No message is delivered by a process more than once.
	
	\item No Creation: If some process q delivers a message m with sender p, then m was previously sent by process p to q.

\end{itemize}

The Reliable Delivery property is simulated in scenario \textit{Pp2pLinkScenario.java}. It let process instance of \textit{LinkPoint.java} to send predevided fixed number of messages (1000) to another instance of \textit{LinkPoint.java}, and \textit{Pp2pSimulationObserver.java} listens on number of messages received at second instnce, and captures in a 
\textit{GlobalView}. GlobalView terminate and report to Simulation at sucessfull receival of all messages.

Although this can be even proven with the standard trasport-level protocol used in the Network model, which is TCP. 
So, As our Perfect Point to Point Link abstraction is build on top of TCP, No Duplication and No creation of messages are also satisfied.

\subsection{Best Effort Broadcast}

\section{Conclusions}

\textit{The report should not be too long ($\approx$
  2-3 pages).}

What have you learnt from the problem presented?
Was it useful?


\end{document}
